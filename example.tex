\documentclass[a4paper,12pt]{article}
\usepackage[spanish]{babel}
\usepackage[utf8x]{inputenc}






%opening
\title{Proyecto fin de carrera}
\author{Matilde Gómez Ojeda}
\date {fecha}


\begin{document}
\maketitle
\tableofcontents




\begin{abstract}

%Este es el resumen
\end{abstract}
\newpage



\section{Introducción}
Texto de la introducción



%\includegraphics[width= 0.75 \textwidth]{lola.jpg}


\newpage
\section{Análisis}


\subsection{¿Qué es el cuaderno del profesor?}
Se trata de un \textbf {cuaderno donde se toman apuntes} de las notas de los exámenes y otras pruebas, datos del alumnado, 
control de asistencia, etc. En definitiva, donde se registra el trabajo que se realiza a lo largo del curso. Es
una herramienta fundamental a la hora de evaluar.
\\Llevar un \textbf {registro de todas las tareas} (realizadas y por hacer), exámenes, actitudes del alumno, observaciones 
durante el proceso de las clases didácticas.
Hacer \textbf {informes individuales y colectivos} de calificaciones, rendimiento académico, faltas de asistencia,
etc., de los alumnos.


\newpage
\section{Diseño}
Texto del diseño

\newpage
\section{Implementación}
Texto de la implementación

\newpage
\section{Anexos}
Texto de la primera seccion

\subsection{subseccion}
Subseccion

\subsubsection{subsubseccion}
texto



\end{document}
